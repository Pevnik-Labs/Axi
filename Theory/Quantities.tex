\documentclass{beamer}
\usepackage[utf8]{inputenc}
\usepackage{babel}
\usepackage{xcolor}
\usepackage{Commands}

\usetheme{Darmstadt}

\title{Quantities}

\begin{document}

\section{Quantities}

\begin{frame}{Quantities}

Quantities: \\
$r, s, q ::= \Zero \pipe \One \pipe \Few \pipe \Many \pipe \Any$

\end{frame}

\begin{frame}{Quantities -- explanation}

\begin{itemize}
  \item $\Zero$ means a resource has been used up and is no longer available.
  \item $\One$ means a resource must be used exactly once.
  \item $\Few$ (pronounced ``few'') means a resource must be used at most once.
  \item $\Many$ (pronounced ``many'') means a resource must be used at least once.
  \item $\Any$ (pronounced ``any'') means no restrictions on usage.
\end{itemize}

\end{frame}

\begin{frame}{Subusage ordering}

$\subusage{r}{s}$ means that a resource with quantity $r$ may be used when quantity $s$ is expected. This ordering is called sub-usaging. The definition below is just the skeleton, the full ordering is the reflexive-transitive closure of it.

\begin{center}
  $\subusage{\Any}{\Few}$ \\
  $\subusage{\Any}{\Many}$ \\
  $\subusage{\Many}{\One}$ \\
  $\subusage{\Few}{\One}$ \\
  $\subusage{\Few}{\Zero}$
\end{center}

\vspace{2em}

We will denote the infimum in this order with $\infqty{r}{s}$ and the supremum (if it exists) with $\supqty{r}{s}$.

\end{frame}

\begin{frame}{Addition}

When we have two quantities of the same resource, we can sum the quantities.

\vspace{2em}

\begin{center}
  $\addqty{\Zero}{r} = r$ \\
  $\addqty{r}{\Zero} = r$ \\
  $\addqty{\Few}{\Few} = \Any$ \\
  $\addqty{\Few}{\Any} = \Any$ \\
  $\addqty{\Any}{\Few} = \Any$ \\
  $\addqty{\Any}{\Any} = \Any$ \\
  $\addqty{\_}{\_} = \Many$
\end{center}

\end{frame}

\begin{frame}{Multiplication}

When we have a quantity $r$ of resource $A$ that contains quantitty $s$ of resource $B$, then we in fact have quantity $\mulqty{r}{s}$ of resource $B$.

\vspace{2em}

\begin{center}
  $\mulqty{\Zero}{r} = \Zero$ \\
  $\mulqty{r}{\Zero} = \Zero$ \\
  $\mulqty{\One}{r} = r$ \\
  $\mulqty{r}{\One} = r$ \\
  $\mulqty{\Few}{\Few} = \Few$ \\
  $\mulqty{\Many}{\Many} = \Many$ \\
  $\mulqty{\_}{\_} = \Any$
\end{center}

\end{frame}

\begin{frame}{The algebra of quantities}

Quantities $\qty$ form a positive ordered commutative semiring with no zero divisors, i.e.:

\begin{itemize}
  \item $(\qty, \addqty{}{}, \Zero)$ is a commutative monoid.
  \item $(\qty, \mulqty{}{}, \One)$ is a commutative monoid.
  \item $\Zero$ annihilates multiplication.
  \item Multiplication distributes over addition.
  \item Addition and multiplication preserve the subusage ordering in both arguments.
  \item If $\addqty{r}{s} = \Zero$, then $r = \Zero$ and $s = \Zero$.
  \item If $\mulqty{r}{s} = \Zero$, then $r = \Zero$ or $s = \Zero$.
\end{itemize}

\end{frame}

\begin{frame}{Infimum}

We can explicitly calculate the infimum as follows.

\begin{center}
  $\infqty{\Zero}{\One} = \Few$ \\
  $\infqty{\Zero}{\Many} = \Any$ \\
  $\infqty{\Zero}{s} = s$ \\
  $\infqty{\One}{\Zero} = \Few$ \\
  $\infqty{\Many}{\Zero} = \Any$ \\
  $\infqty{r}{\Zero} = r$ \\
  $\infqty{r}{s} = \mulqty{r}{s}$
\end{center}

\end{frame}

\begin{frame}{Complement}

Complement of $r$ is the quantity on the opposite side of the diamond. The complement of $\Zero$ is $\Zero$.

\begin{center}
  $\complementqty{\Zero} = \Zero$ \\
  $\complementqty{\One} = \Any$ \\
  $\complementqty{\Few} = \Many$ \\
  $\complementqty{\Many} = \Few$ \\
  $\complementqty{\Any} = \One$
\end{center}

\end{frame}

\begin{frame}{Supremum}

We can explicitly calculate the supremum as follows.

\begin{center}
  $\supqty{\Zero}{\One} = \texttt{undefined}$ \\
  $\supqty{\Zero}{\Many} = \texttt{undefined}$ \\
  $\supqty{\One}{\Zero} = \texttt{undefined}$ \\
  $\supqty{\One}{\Many} = \texttt{undefined}$ \\
  $\supqty{r}{s} = \complementqty{(\mulqty{\complementqty{r}}{\complementqty{s}})}$
\end{center}

\end{frame}

\begin{frame}{Subtraction}

$\subqty{r}{s} = \inf \{q \in \qty \pipe \subusage{r}{\addqty{q}{s}} \}$, where the $\inf$ is taken according to the subusage ordering. Explicitly:

\vspace{2em}

\begin{table}[ht]
  \centering
  \begin{tabular}{|c|c|c|c|c|c|}
  \hline
  $\subqty{r}{s}$ & $\Zero$ & $\One$  & $\Few$  & $\Many$ & $\Any$  \\ \hline
  $\Zero$             & $\Zero$ &         &         &         &         \\ \hline
  $\One$              & $\One$  & $\Zero$ &         &         &         \\ \hline
  $\Few$              & $\Few$  & $\Zero$ & $\Zero$ &         &         \\ \hline
  $\Many$             & $\Many$ & $\Any$  & $\Many$ & $\Any$  & $\Many$ \\ \hline
  $\Any$              & $\Any$  & $\Any$  & $\Any$  & $\Any$  & $\Any$  \\ \hline
  \end{tabular}
\end{table}

\end{frame}

\begin{frame}{Subtraction order}

$\lesubqty{r}{s}$ holds when $\subqty{s}{r}$ is defined.

\vspace{2em}

Explicitly: $\Zero \lesubqty{}{} \One \lesubqty{}{} \Few \lesubqty{}{} \Many \lesubqty{}{} \Any \lesubqty{}{} \Many$

\end{frame}

\begin{frame}{Decrementation order}

$\ledecrqty{r}{s}$ holds when $r = \subqty{s}{\One}$.

\begin{center}
  $\ledecrqty{\Any}{\Any}$ \\
  $\ledecrqty{\Any}{\Many}$ \\
  $\ledecrqty{\Zero}{\One}$ \\
  $\ledecrqty{\Zero}{\Few}$
\end{center}

\end{frame}

\begin{frame}{Division with remainder}

$\divmodqty{r}{s} = \sup_{a \in \qty} \inf_{b \in \qty} \{(a, b) \pipe r = \addqty{\mulqty{a}{s}}{b} \}$, where the $\sup$ and $\inf$ are taken according to the subtraction order. Note that $\divmodqty{r}{s} = a$ means that $b = \Zero$. Explicitly:

\vspace{2em}

\begin{table}[ht]
  \centering
  \begin{tabular}{|c|c|c|c|c|c|}
  \hline
  $\divmodqty{r}{s}$  & $\Zero$         & $\One$  & $\Few$          & $\Many$         & $\Any$          \\ \hline
  $\Zero$             & $\Any$          & $\Zero$ & $\Zero$         & $\Zero$         & $\Zero$         \\ \hline
  $\One$              & $(\Any, \One)$  & $\One$  & $(\Zero, \One)$ & $(\Zero, \One)$ & $(\Zero, \One)$ \\ \hline
  $\Few$              & $(\Any, \Few)$  & $\Few$  & $\Few$          & $(\Zero, \Few)$ & $(\Zero, \Few)$ \\ \hline
  $\Many$             & $(\Any, \Many)$ & $\Many$ & $(\Any, \One)$  & $\Many$         & $(\Any, \One)$  \\ \hline
  $\Any$              & $(\Any, \Any)$  & $\Any$  & $\Any$          & $\Any$          & $\Any$          \\ \hline
  \end{tabular}
\end{table}

\end{frame}

\begin{frame}{Trait ordering}

The trait ordering is similar to the subusage ordering, but $\Zero$ is above $\One$. Explicitly:

\begin{center}
  $\letraitqty{\Any}{\Few}$ \\
  $\letraitqty{\Any}{\Many}$ \\
  $\letraitqty{\Few}{\One}$ \\
  $\letraitqty{\Many}{\One}$ \\
  $\letraitqty{\One}{\Zero}$
\end{center}

\end{frame}

\begin{frame}{Trait division}

$\divtraitqty{r}{s} = \sup \{q \in \qty \pipe \subusage{\mulqty{q}{s}}{r} \}$, where the $\sup$ is taken according to the trait ordering. Explicitly:

\vspace{2em}

\begin{table}[ht]
  \centering
  \begin{tabular}{|c|c|c|c|c|c|}
  \hline
  $\divtraitqty{r}{s}$ & $\Zero$ & $\One$  & $\Few$   & $\Many$ & $\Any$  \\ \hline
  $\Zero$              & $\Zero$ & $\Zero$ & $\Zero$  & $\Zero$ & $\Zero$ \\ \hline
  $\One$               &         & $\One$  & $\One$   & $\One$  & $\One$  \\ \hline
  $\Few$               &         & $\Few$  & $\One$   & $\Few$  & $\One$  \\ \hline
  $\Many$              &         & $\Many$ & $\Many$  & $\One$  & $\One$  \\ \hline
  $\Any$               &         & $\Any$  & $\Many$  & $\Few$  & $\One$  \\ \hline
  \end{tabular}
\end{table}

\end{frame}

\begin{frame}{Arithmetic order}

The arithmetic order on quantities is $\Zero \learithqty \One \learithqty \Few \learithqty \Many \learithqty \Any$. The idea is to compare the quantities by how ``big'' they are.

\end{frame}

\end{document}