\documentclass{beamer}
\usepackage[utf8]{inputenc}
\usepackage{babel}
\usepackage{xcolor}
\usepackage{Commands}

\usetheme{Darmstadt}

\title{Quantities}

\begin{document}

\section{Quantities}

\begin{frame}{Quantities}

Quantities: \\
$r ::= \Zero \pipe \One \pipe \Few \pipe \Many \pipe \Any$

\vspace{2em}

$\Any$ is the default quantity, so when there's nothing to indicate quantity, it means it's $\Any$.

\end{frame}

\begin{frame}{Quantities -- explanation}

\begin{itemize}
  \item $\Zero$ means a resource has been used up and is no longer available.
  \item $\One$ means a resource must be used exactly once.
  \item $\Few$ (pronounced ``few'') means a resource must be used at most once.
  \item $\Many$ (pronounced ``many'') means a resource must be used at least once.
  \item $\Any$ (pronounced ``any'') means no restrictions on usage.
\end{itemize}

\end{frame}

\begin{frame}{Subusage ordering}

$\subusage{r_1}{r_2}$ means that a resource with quantity $r_1$ may be used when quantity $r_2$ is expected. This ordering is called sub-usaging. The definition below is just the skeleton, the full ordering is the reflexive-transitive closure of it.

\begin{center}
  $\subusage{\Any}{\Few}$ \\
  $\subusage{\Any}{\Many}$ \\
  $\subusage{\Many}{\One}$ \\
  $\subusage{\Few}{\One}$ \\
  $\subusage{\Few}{\Zero}$
\end{center}

\vspace{2em}

We will denote the greatest lower bound in this order with $\glbqty{r_1}{r_2}$ and the least upper bound (if it exists) with $\lubqty{r_1}{r_2}$.

\end{frame}

\begin{frame}{Addition of quantities}

When we have two quantities of the same resource, we can sum the quantities.

\vspace{2em}

\begin{center}
  $\addqty{\Zero}{r} = r$ \\
  $\addqty{r}{\Zero} = r$ \\
  $\addqty{\Few}{\Few} = \Any$ \\
  $\addqty{\Few}{\Any} = \Any$ \\
  $\addqty{\Any}{\Few} = \Any$ \\
  $\addqty{\Any}{\Any} = \Any$ \\
  $\addqty{\_}{\_} = \Many$
\end{center}

\end{frame}

\begin{frame}{Multiplication of quantities}

When we have a quantity $r_1$ of resource $A$ that contains quantitty $r_2$ of resource $B$, then we in fact have quantity $\mulqty{r_1}{r_2}$ of resource $B$.

\vspace{2em}

\begin{center}
  $\mulqty{\Zero}{r} = \Zero$ \\
  $\mulqty{r}{\Zero} = \Zero$ \\
  $\mulqty{\One}{r} = r$ \\
  $\mulqty{r}{\One} = r$ \\
  $\mulqty{\Few}{\Few} = \Few$ \\
  $\mulqty{\Many}{\Many} = \Many$ \\
  $\mulqty{\_}{\_} = \Any$
\end{center}

\end{frame}

\begin{frame}{The algebra of quantities}

Quantities $\qty$ form a positive ordered commutative semiring with no zero divisors, i.e.:

\begin{itemize}
  \item $(\qty, \addqty{}{}, \Zero)$ is a commutative monoid.
  \item $(\qty, \mulqty{}{}, \One)$ is a commutative monoid.
  \item $\Zero$ annihilates multiplication.
  \item Multiplication distributes over addition.
  \item Addition and multiplication preserve the subusage ordering in both arguments.
  \item If $\addqty{r_1}{r_2} = \Zero$, then $r_1 = \Zero$ and $r_2 = \Zero$.
  \item If $\mulqty{r_1}{r_2} = \Zero$, then $r_1 = \Zero$ or $r_2 = \Zero$.
\end{itemize}

\end{frame}

\begin{frame}{Subtraction of quantities}

$\subqty{r_1}{r_2}$ is the least $r'$ such that $\subusage{r_1}{\addqty{r'}{r_2}}$.

\vspace{2em}

\begin{table}[ht]
  \centering
  \begin{tabular}{|c|c|c|c|c|c|}
  \hline
  $\subqty{r_1}{r_2}$ & $\Zero$ & $\One$  & $\Few$  & $\Many$ & $\Any$  \\ \hline
  $\Zero$             & $\Zero$ &         &         &         &         \\ \hline
  $\One$              & $\One$  & $\Zero$ &         &         &         \\ \hline
  $\Few$              & $\Few$  & $\Zero$ & $\Zero$ &         &         \\ \hline
  $\Many$             & $\Many$ & $\Any$  & $\Many$ & $\Any$  & $\Many$ \\ \hline
  $\Any$              & $\Any$  & $\Any$  & $\Any$  & $\Any$  & $\Any$  \\ \hline
  \end{tabular}
\end{table}

\end{frame}

\begin{frame}{Subtraction order on quantities}

$\lesubqty{r_1}{r_2}$ holds when $\subqty{r_2}{r_1}$ is defined.

\vspace{2em}

Explicitly: $\Zero \lesubqty{}{} \One \lesubqty{}{} \Few \lesubqty{}{} \Many \lesubqty{}{} \Any \lesubqty{}{} \Many$

\end{frame}

\begin{frame}{Decrementation order on quantities}

$\ledecqty{r_1}{r_2}$ holds when $\subqty{r_2}{\One} = r_1$.

\begin{center}
  $\infrule{}{\ledecqty{\Any}{\Any}}$

  \vspace{2em}

  $\infrule{}{\ledecqty{\Any}{\Many}}$

  \vspace{2em}

  $\infrule{}{\ledecqty{\Zero}{\One}}$

  \vspace{2em}

  $\infrule{}{\ledecqty{\Zero}{\Few}}$
\end{center}

\end{frame}

\begin{frame}{Arithmetic order on quantities}

The arithmetic order on quantities is $\Zero \learithqty \One \learithqty \Few \learithqty \Many \learithqty \Any$. The idea is to compare the quantities by how ``big'' they are.

\end{frame}

\begin{frame}{Trait-checking Division}

$\divqty{r_1}{r_2} = \sup \{s \in \qty \pipe \subusage{\mulqty{s}{r_2}}{r_1} \}$

\begin{table}[ht]
  \centering
  \begin{tabular}{|c|c|c|c|c|c|}
  \hline
  $\divqty{r_1}{r_2}$ & $\Zero$ & $\One$  & $\Few$   & $\Many$ & $\Any$  \\ \hline
  $\Zero$             & $\Zero$ & $\Zero$ & $\Zero$  & $\Zero$ & $\Zero$ \\ \hline
  $\One$              &         & $\One$  & $\One$   & $\One$  & $\One$  \\ \hline
  $\Few$              &         & $\Few$  & $\One$   & $\Few$  & $\One$  \\ \hline
  $\Many$             &         & $\Many$ & $\Many$  & $\One$  & $\One$  \\ \hline
  $\Any$              &         & $\Any$  & $\Many$  & $\Few$  & $\One$  \\ \hline
  \end{tabular}
\end{table}

\end{frame}

\begin{frame}{Division with remainder}

$\divqty{r_1}{r_2} = (a, b)$ when $r_1 = \addqty{\mulqty{a}{r_2}}{b}$, with $a$ as large as possible and $b$ being as small as possible according to the subtraction order. Note that $\divqty{r_1}{r_2} = a$ means that $b = \Zero$.

\vspace{2em}

\begin{table}[ht]
  \centering
  \begin{tabular}{|c|c|c|c|c|c|}
  \hline
  $\divqty{r_1}{r_2}$ & $\Zero$         & $\One$  & $\Few$          & $\Many$         & $\Any$          \\ \hline
  $\Zero$             & $\Any$          & $\Zero$ & $\Zero$         & $\Zero$         & $\Zero$         \\ \hline
  $\One$              & $(\Any, \One)$  & $\One$  & $(\Zero, \One)$ & $(\Zero, \One)$ & $(\Zero, \One)$ \\ \hline
  $\Few$              & $(\Any, \Few)$  & $\Few$  & $\Few$          & $(\Zero, \Few)$ & $(\Zero, \Few)$ \\ \hline
  $\Many$             & $(\Any, \Many)$ & $\Many$ & $(\Any, \One)$  & $\Many$         & $(\Any, \One)$  \\ \hline
  $\Any$              & $(\Any, \Any)$  & $\Any$  & $\Any$          & $\Any$          & $\Any$          \\ \hline
  \end{tabular}
\end{table}

\end{frame}

\end{document}