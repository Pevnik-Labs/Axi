\documentclass{beamer}
\usepackage[utf8]{inputenc}
\usepackage{babel}
\usepackage{xcolor}
\usepackage{AxiCommands}

\usetheme{Darmstadt}

\title{Poor Man's Axi: Algorithmic Version}
\author{Wojciech Kołowski}
\date{}

\begin{document}

\frame{\titlepage}

\section{Semantics}

\begin{frame}{Values (cbv)}

\begin{center}
  $\infrule{}{\val{\fun{x}{e}}}$

  \vspace{2em}

  $\infrule{\val{v_1} \quad \val{v_2}}{\val{\pair{v_1}{v_2}}}$

  \vspace{2em}

  $\infrule{\val{v}}{\val{\inl{v}}}$ \quad
  $\infrule{\val{v}}{\val{\inr{v}}}$ \quad

  \vspace{2em}

  $\infrule{}{\val{\unit}}$
\end{center}

\vspace{2em}

Values are the final results of computation. Note that a function is a value whether or not its body is. Other values are pairs of values, values injected into a sum on the left or right, and unit.

\end{frame}

\begin{frame}{Big-step semantics (cbv)}

\begin{center}
  $\infrule{}{\bigstep{\fun{x}{e}}{\fun{x}{e}}}$ \quad
  $\infrule{\bigstep{e_1}{\fun{x}{e}} \quad \bigstep{e_2}{v} \quad \bigstep{\subst{e}{x}{v}}{v'}}{\bigstep{\app{e_1}{e_2}}{v'}}$

  \vspace{2em}

  $\infrule{\bigstep{e_1}{v_1} \quad \bigstep{e_2}{v_2}}{\bigstep{\pair{e_1}{e_2}}{\pair{v_1}{v_2}}}$ \quad
  $\infrule{\bigstep{e}{\pair{v_1}{v_2}}}{\bigstep{\outl{e}}{v_1}}$ \quad
  $\infrule{\bigstep{e}{\pair{v_1}{v_2}}}{\bigstep{\outr{e}}{v_2}}$ \quad

  \vspace{2em}

  $\infrule{\bigstep{e}{v}}{\bigstep{\inl{e}}{\inl{v}}}$ \quad
  $\infrule{\bigstep{e}{v}}{\bigstep{\inr{e}}{\inr{v}}}$ \quad

  \vspace{2em}

  $\infrule{\bigstep{e}{\inl{v}} \quad \bigstep{\app{f}{v}}{v'}}{\bigstep{\case{e}{f}{g}}{v'}}$ \quad
  $\infrule{\bigstep{e}{\inr{v}} \quad \bigstep{\app{g}{v}}{v'}}{\bigstep{\case{e}{f}{g}}{v'}}$

  \vspace{2em}

  $\infrule{}{\bigstep{\unit}{\unit}}$ \quad
\end{center}

\end{frame}

\begin{frame}{Small-step semantics (cbv) -- basic rules}

\begin{center}
  $\infrule{\val{v}}{\smallstep{\app{(\fun{x}{e})}{v}}{\subst{e}{x}{v}}}$

  \vspace{2em}

  $\infrule{\val{v_1} \quad \val{v_2}}{\smallstep{\outl{\pair{v_1}{v_2}}}{v_1}}$ \quad
  $\infrule{\val{v_1} \quad \val{v_2}}{\smallstep{\outr{\pair{v_1}{v_2}}}{v_2}}$

  \vspace{2em}

  $\infrule{\val{v}}{\smallstep{\case{(\inl{v})}{f}{g}}{\app{f}{v}}}$

  \vspace{2em}

  $\infrule{\val{v}}{\smallstep{\case{(\inr{v})}{f}{g}}{\app{g}{v}}}$

\end{center}

\end{frame}

\begin{frame}{Small-step semantics (cbv) -- boring rules}

\begin{center}
  $\infrule{\smallstep{e_1}{e_1'}}{\smallstep{\app{e_1}{e_2}}{\app{e_1'}{e_2}}}$ \quad
  $\infrule{\val{v_1} \quad \smallstep{e_2}{e_2'}}{\smallstep{\app{v_1}{e_2}}{\app{v_1}{e_2'}}}$

  \vspace{2em}

  $\infrule{\smallstep{e_1}{e_1'}}{\smallstep{\pair{e_1}{e_2}}{\pair{e_1'}{e_2}}}$ \quad
  $\infrule{\val{v_1} \quad \smallstep{e_2}{e_2'}}{\smallstep{\pair{v_1}{e_2}}{\pair{v_1}{e_2'}}}$

  \vspace{2em}

  $\infrule{\smallstep{e}{e'}}{\smallstep{\outl{e}}{\outl{e'}}}$ \quad
  $\infrule{\smallstep{e}{e'}}{\smallstep{\outr{e}}{\outr{e'}}}$

  \vspace{2em}

  $\infrule{\smallstep{e}{e'}}{\smallstep{\inl{e}}{\inl{e'}}}$ \quad
  $\infrule{\smallstep{e}{e'}}{\smallstep{\inr{e}}{\inr{e'}}}$

  \vspace{2em}

  $\infrule{\smallstep{e}{e'}}{\smallstep{\case{e}{f}{g}}{\case{e'}{f}{g}}}$

\end{center}

\end{frame}

\begin{frame}{Example semantics -- explanations}

$\bigstep{e}{v}$ should be read ``term $e$ evalutes to value $v$''. It describes computation in a coarse-grained manner, telling us what is the result of evaluating each term. If $\bigstep{e}{v}$, then $\val{v}$.

\vspace{2em}

$\smallstep{e}{e'}$ should be read ``term $e$ reduces to term $e'$''. It describes computation in a fine-grained manner, telling us what can happen in a single computation step. To actually describe computation fully in this style, we need to take the transitive closure of this relation, written $\smallsteps{e}{v}$.

\vspace{2em}

Both presented semantics are call-by-value, i.e. we evaluate a function's argument before performing a substitution. They are equivalent, i.e. $\bigstep{e}{v}$ if and only if $\smallsteps{e}{v}$

\end{frame}

\newcommand{\whnf}[1]{#1\ \texttt{whnf}}

\begin{frame}{Weak head normal forms}

\begin{center}
  $\infrule{}{\whnf{x}}$

  \vspace{2em}

  $\infrule{}{\whnf{\fun{x}{e}}}$

  \vspace{2em}

  $\infrule{}{\whnf{\pair{e_1}{e_2}}}$

  \vspace{2em}

  $\infrule{}{\whnf{\inl{e}}}$ \quad
  $\infrule{}{\whnf{\inr{e}}}$ \quad

  \vspace{2em}

  $\infrule{}{\whnf{\unit}}$
\end{center}

\end{frame}

\begin{frame}{Weak head normal forms grammar}

Weak head normal forms: \\
$e ::=$ \\
\qquad $n \pipe \fun{x}{e} \pipe$ \\
\qquad $\pair{e_1}{e_2} \pipe$ \\
\qquad $\inl{e} \pipe \inr{e} \pipe$ \\
\qquad $\unit \pipe \elimEmpty{e}$

\vspace{2em}

Neutral forms: \\
$n ::=$ \\
\qquad $x \pipe \app{n}{e} \pipe$ \\
\qquad $\outl{n} \pipe \outr{n} \pipe$ \\
\qquad $\case{n}{e_1}{e_2} \pipe$ \\

\end{frame}

\newcommand{\smallstepwhnf}[2]{#1 \longrightarrow_{\texttt{whnf}} #2}

\begin{frame}{Whnf reduction -- basic rules}

\begin{center}
  $\infrule{}{\smallstepwhnf{\app{(\fun{x}{e_1})}{e_2}}{\subst{e_1}{x}{e_2}}}$

  \vspace{2em}

  $\infrule{}{\smallstepwhnf{\outl{\pair{e_1}{e_2}}}{e_1}}$ \quad
  $\infrule{}{\smallstepwhnf{\outr{\pair{e_1}{e_2}}}{e_2}}$

  \vspace{2em}

  $\infrule{}{\smallstepwhnf{\case{(\inl{e})}{f}{g}}{\app{f}{e}}}$

  \vspace{2em}

  $\infrule{}{\smallstepwhnf{\case{(\inr{e})}{f}{g}}{\app{g}{e}}}$

\end{center}

\end{frame}

\begin{frame}{Whnf reduction -- boring rules}

\begin{center}
  $\infrule{\smallstepwhnf{e_1}{e_1'}}{\smallstepwhnf{\app{e_1}{e_2}}{\app{e_1'}{e_2}}}$ \quad
  $\infrule{\whnf{e_1} \quad \smallstepwhnf{e_2}{e_2'}}{\smallstepwhnf{\app{e_1}{e_2}}{\app{e_1}{e_2'}}}$

  \vspace{2em}

  $\infrule{\smallstepwhnf{e}{e'}}{\smallstepwhnf{\outl{e}}{\outl{e'}}}$ \quad
  $\infrule{\smallstepwhnf{e}{e'}}{\smallstepwhnf{\outr{e}}{\outr{e'}}}$

  \vspace{2em}

  $\infrule{\smallstepwhnf{e}{e'}}{\smallstepwhnf{\case{e}{f}{g}}{\case{e'}{f}{g}}}$

\end{center}

\end{frame}

\section{Algorithmic computational equality}

\newcommand{\fullcheckcompeq}[4]{#1 \vdash #2 \equiv #3 \mathcolor{blue}{\Leftarrow} #4}
\newcommand{\checkcompeq}[3]{\fullcheckcompeq{\Gamma}{#1}{#2}{#3}}

\newcommand{\fullinfercompeq}[4]{#1 \vdash #2 \equiv #3 \mathcolor{red}{\Rightarrow} #4}
\newcommand{\infercompeq}[3]{\fullinfercompeq{\Gamma}{#1}{#2}{#3}}

\newcommand{\fullcheckcompeqwhnf}[4]{#1 \vdash #2 \equiv #3 \mathcolor{blue}{\Leftarrow}_{\texttt{whnf}} #4}
\newcommand{\checkcompeqwhnf}[3]{\fullcheckcompeqwhnf{\Gamma}{#1}{#2}{#3}}

\newcommand{\fullinfercompeqwhnf}[4]{#1 \vdash #2 \equiv #3 \mathcolor{red}{\Rightarrow}_{\texttt{whnf}} #4}
\newcommand{\infercompeqwhnf}[3]{\fullinfercompeqwhnf{\Gamma}{#1}{#2}{#3}}

\begin{frame}{Algorithmic computational equality 0}

\begin{center}
  $\infrule[Var]{\sidecond{(x : A) \in \Gamma}}{\infercompeq{x}{x}{A}}$

  \vspace{2em}

  %$\infrule[Annot]{\checkcompeq{e}{e'}{A}}{\infercompeq{\annot{e}{A}}{ź}{A}}$

  \vspace{2em}

  $\infrule[Sub]{\infercompeq{e}{e'}{B} \quad \sidecond{A = B}}{\checkcompeq{e}{e'}{A}}$

  \vspace{2em}

  $\infrule{\smallstepwhnf{e_1}{e_1'} \quad \smallstepwhnf{e_2}{e_2'} \quad \checkcompeq{e_1'}{e_2'}{A}}{\checkcompeqwhnf{e_1}{e_2}{A}}$

  \vspace{2em}

  $\infrule{\smallstepwhnf{e_1}{e_1'} \quad \smallstepwhnf{e_2}{e_2'} \quad \infercompeq{e_1'}{e_2'}{A}}{\infercompeqwhnf{e_1}{e_2}{A}}$
\end{center}

\end{frame}

\begin{frame}{Algorithmic computational equality 1}

\begin{center}
  $\infrule{\fullcheckcompeqwhnf{\extend{\Gamma}{x}{A}}{e_1}{e_2}{B}}{\checkcompeq{\fun{x}{e_1}}{\fun{x}{e_2}}{\Fun{A}{B}}}$

  \vspace{2em}

  $\infrule{\infercompeq{n_1}{n_2}{\Fun{A}{B}} \quad \checkcompeqwhnf{e_1}{e_2}{A}}{\infercompeq{\app{n_1}{e_1}}{\app{n_2}{e_2}}{B}}$

  \vspace{2em}

  $\infrule{\checkcompeqwhnf{a_1}{a_2}{A} \quad \checkcompeqwhnf{b_1}{b_2}{B}}{\checkcompeq{\pair{a_1}{b_1}}{\pair{a_2}{b_2}}{\Prod{A}{B}}}$

  \vspace{2em}

  $\infrule{\infercompeq{n_1}{n_2}{\Prod{A}{B}}}{\infercompeq{\outl{n_1}}{\outl{n_2}}{A}}$ \quad
  $\infrule{\infercompeq{n_1}{n_2}{\Prod{A}{B}}}{\infercompeq{\outr{n_1}}{\outr{n_2}}{B}}$
\end{center}

\end{frame}

\begin{frame}{Algorithmic computational equality 2}

\begin{center}
  $\infrule{\checkcompeqwhnf{e_1}{e_2}{A}}{\checkcompeq{\inl{e_1}}{\inl{e_2}}{\Sum{A}{B}}}$ \quad
  $\infrule{\checkcompeqwhnf{e_1}{e_2}{B}}{\checkcompeq{\inr{e_1}}{\inr{e_2}}{\Sum{A}{B}}}$

  \vspace{2em}

  $\infrule{\infercompeq{n_1}{n_2}{\Sum{A}{B}} \quad \begin{array}{c} \checkcompeqwhnf{f_1}{f_2}{\Fun{A}{C}} \\ \checkcompeqwhnf{g_1}{g_2}{\Fun{B}{C}} \end{array}}{\infercompeq{\case{n_1}{f_1}{g_1}}{\case{n_2}{f_2}{g_2}}{C}}$

  \vspace{2em}

  $\infrule{}{\checkcompeq{\unit}{\unit}{\Unit}}$ \quad
  $\infrule{\checkcompeq{e_1}{e_2}{\Empty}}{\checkcompeq{\elimEmpty[e]}{\elimEmpty[e']}{A}}$
\end{center}

\end{frame}

\begin{frame}{Uniqueness rules (asymmetric, contraction-like)}

\begin{center}
  $\infrule[Fun-Uniq]{\check{f}{\Fun{A}{B}}}{\checkcompeq{f}{\fun{x}{\app{f}{x}}}{\Fun{A}{B}}}$

  \vspace{2em}

  $\infrule[Prod-Uniq]{\check{e}{\Prod{A}{B}}}{\checkcompeq{e}{\pair{\outl{e}}{\outr{e}}}{\Prod{A}{B}}}$

  \vspace{2em}

  $\infrule[Unit-Uniq]{\check{e}{\Unit}}{\checkcompeq{e}{\unit}{\Unit}}$
\end{center}

\end{frame}

\end{document}